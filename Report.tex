\documentclass[conference]{IEEEtran}
\usepackage{cite}
\usepackage{amsmath,amssymb,amsfonts}
\usepackage{algorithmic}
\usepackage{graphicx}
\usepackage{textcomp}
\usepackage{xcolor}
\usepackage{placeins}
\def\BibTeX{{\rm B\kern-.05em{\sc i\kern-.025em b}\kern-.08em
    T\kern-.1667em\lower.7ex\hbox{E}\kern-.125emX}}

\title{NSL-KDD-Network-Security-Report}
\author{\IEEEauthorblockN{Rojina Deuja}
\IEEEauthorblockA{\textit{Department of Computer Science and Engineering} \\
\textit{University of Nebraska-Lincoln}\\
Lincoln, United States \\
rojinadeuja33g@gmail.com}}
\date{November 2020}

\begin{document}

\maketitle

\begin{abstract}

\end{abstract}

\begin{IEEEkeywords}
\end{IEEEkeywords}

\section{Introduction}
Network Security is an ever-expanding and highly demanding research areas in the field of Information Technology. Network security is primarily concerned with implementing a set of rules, configurations or systems that are designed to enhance the protection of critical data in any communication network. Whether it be a home network or a business network, there is always the chance of the system could be exploited if not properly secured. An established network security system can help prevent the loss, theft or unauthorized access of sensitive information.

An intrusion is an attempt to compromise Confidentiality, Integrity and Availability (CIA), or to bypass the security mechanisms of a computer or network \cite{b1}. Intrusion detection is the task of monitoring various events that occur in a computer system or a communication network, and analyzing them for signs of intrusions. Intrusion detection methodologies are generally classified into three main categories: Signature-based Detection (SD), Anomaly-based Detection (AD) and Stateful Protocol Analysis (SPA). Anomaly-based Detection (AD) identifies anomalous events in the network system. An anomaly is any event that deviates from normal/known behavior. These anomalies can be identified by monitoring or evaluation regular activities, connections, hosts or users in the network over a period of time \cite{b2}.

\section{Literature Review}
\section{Dataset}
\section{Implementation}
\section{Results}
\section{Conclusion}
\section*{Acknowledgment}

\begin{thebibliography}{00}
\bibitem{b1} R. Bace, P. Mell, "Intrusion detection systems,National Institute of Standards and
Technology (NIST)`` , Technical Report, pp. 800-31, 2001.
\bibitem{b2} H. Liao, C. Lin, Y. Lin and K.Tung, "Intrusion detection system: A comprehensive review``, Journal of Network and Computer Applications, vol. 36, pp.16-24, 2013.
\end{thebibliography}

\end{document}
