\documentclass[conference]{IEEEtran}
\usepackage{cite}
\usepackage{amsmath,amssymb,amsfonts}
\usepackage{algorithmic}
\usepackage{graphicx}
\usepackage{textcomp}
\usepackage{xcolor}
\usepackage{placeins}
\def\BibTeX{{\rm B\kern-.05em{\sc i\kern-.025em b}\kern-.08em
    T\kern-.1667em\lower.7ex\hbox{E}\kern-.125emX}}

\title{NSL-KDD Network Security Report}
\author{\IEEEauthorblockN{Rojina Deuja}
\IEEEauthorblockA{\textit{Department of Computer Science and Engineering} \\
\textit{University of Nebraska-Lincoln}\\
Lincoln, United States \\
rojinadeuja33g@gmail.com}}
\date{November 2020}

\begin{document}

\maketitle

\begin{abstract}
\end{abstract}

\begin{IEEEkeywords}
\end{IEEEkeywords}

\section{Introduction}
Network Security is an ever-expanding and highly demanding research areas in the field of Information Technology. Network security is primarily concerned with implementing a set of rules, configurations or systems that are designed to enhance the protection of critical data in any communication network. Whether it be a home network or a business network, there is always the chance of the system could be exploited if not properly secured. An established network security system can help prevent the loss, theft or unauthorized access of sensitive information.

An intrusion is an attempt to compromise Confidentiality, Integrity and Availability (CIA), or to bypass the security mechanisms of a computer or network \cite{b1}. Intrusion detection is the task of monitoring various events that occur in a computer system or a communication network, and analyzing them for signs of intrusions. Intrusion detection methodologies are generally classified into three main categories: Signature-based Detection (SD), Anomaly-based Detection (AD) and Stateful Protocol Analysis (SPA). Anomaly-based Detection (AD), also known as Behavior-based Detection identifies anomalous events in the network system. An anomaly is any event that deviates from normal/known behavior. These anomalies can be identified by monitoring or evaluation regular activities, connections, hosts or users in the network over a period of time \cite{b2}.

\section{Literature Review}
\section{Dataset}
NSL-KDD dataset was released as a new and improved version of KDD'99 dataset by the University of New Brunswick. It is used as a benchmark data to evaluate various intrusion detection methods on modern-day network systems.
The dataset contains of four different subsets: KDDTrain+, KDDTest+, KDDTrain+\_20Percentand KDDTest+\_20Percent. The first two subsets contain all of the train and test sets while the latter subsets contain only 20\% of the train and test sets respectively.
For the purpose of this experiment, we will use the KDDTrain+ and KDDTest+ datasets and refer to them as train and test sets respectively.
The dataset consists of internet traffic records encountered by a real Intrusion Detection System.

There are 43 attributes in total, where 42 are the input features, `Score' is an field denoting the difficulty of classification and a binary class `Label' that classifies the traffic input into any one of the two labels: Normal or Attack. There are four distinct types of attacks that can be identified: Denial of Service (DoS), Probe, User to Root (U2R) and Remote to Local (R2L).

\section{Intrusion Detection}
There are various techniques that can be applied for Intrusion Detection. Liao et al. \cite{b2} identify three subclasses of intrusion detection methodologies: Signature-based Detection (SD), Anomaly-based Detection (AD) and Stateful Protocol Analysis (SPA). For our intrusion detection task, we will use a Signature-based Detection technique. A signature can be described as a pattern that denotes a known attack or threat. Signature-base Detection is used to identify such patterns by comparing them against normal or expected behavior. Since the knowledge gathered from prior attacks and system vulnerabilities are utilized to find more possibly harmful behavior, SD is also known as Knowledge-based Detection or Misuse Detection. It is one of the simplest and most effective method to detect any forms of anticipated attacks.

\section{Exploratory Data Analysis}
We carry out various data analysis tasks on the NSL-KDD dataset in order to help us understand the data better. There are 43 columns in the dataset, out of which the `attack' field contains the type of attacks detected in the network. There are 22 different types of attacks and a \emph{normal} value for no attack. The types of network attacks and their occurrence in the dataset is shown in Table \ref{tab1}. The most common types of attacks can be seen in the chart shown in Fig \ref{fig1}

\begin{table}[htbp]
\caption{Types of Network Attacks}
\begin{center}
\begin{tabular}{|c|c|}
\hline
Type of Attack & Count \\
\hline
normal (None)	&	67342	\\
neptune	&	41214	\\
satan	&	3633	\\
ipsweep	&	3599	\\
portsweep	&	2931	\\
smurf	&	2646	\\
nmap	&	1493	\\
back	&	956	\\
teardrop	&	892	\\
warezclient	&	890	\\
pod	&	201	\\
guess\_passwd	&	53	\\
buffer\_overflow	&	30	\\
warezmaster	&	20	\\
land	&	18	\\
imap	&	11	\\
rootkit	&	10	\\
loadmodule	&	9	\\
ftp\_write	&	8	\\
multihop	&	7	\\
phf	&	4	\\
perl	&	3	\\
spy	&	2	\\
\hline
\end{tabular}
\label{tab1}
\end{center}
\end{table}

\begin{figure}[htbp]
\centerline{\includegraphics[height= 170 pt, width=0.50\textwidth]{Fig-Frequency-of-Attacks.PNG}}
\caption{Top 10 Most Frequent Attacks}
\label{fig1}
\end{figure}


\section{Implementation}

\section{Results}

\section{Conclusion}

\section*{Acknowledgment}

\begin{thebibliography}{00}
\bibitem{b1} R. Bace, P. Mell, ``Intrusion detection systems,National Institute of Standards and
Technology (NIST)" , Technical Report, pp. 800-31, 2001.
\bibitem{b2} H. Liao, C. Lin, Y. Lin and K.Tung, ``Intrusion detection system: A comprehensive review", Journal of Network and Computer Applications, vol. 36, pp.16-24, 2013.
\end{thebibliography}

\end{document}
