\documentclass[conference]{IEEEtran}
\usepackage{cite}
\usepackage{amsmath,amssymb,amsfonts}
\usepackage{algorithmic}
\usepackage{graphicx}
\usepackage{textcomp}
\usepackage{xcolor}
\usepackage{placeins}
\def\BibTeX{{\rm B\kern-.05em{\sc i\kern-.025em b}\kern-.08em
    T\kern-.1667em\lower.7ex\hbox{E}\kern-.125emX}}

\title{NSL-KDD-Network-Security-Report}
\author{\IEEEauthorblockN{Rojina Deuja}
\IEEEauthorblockA{\textit{Department of Computer Science and Engineering} \\
\textit{University of Nebraska-Lincoln}\\
Lincoln, United States \\
rojinadeuja33g@gmail.com}}
\date{November 2020}

\begin{document}

\maketitle

\maketitle

\begin{abstract}
\end{abstract}

\begin{IEEEkeywords}
\end{IEEEkeywords}

\section{Introduction}
Network Security is is an ever expanding and highly demanding research areas in the field of Information Technology. Network security is primarily concerned with implementing a set of rules, configurations or systems that are designed to enhance the protection of critical data in any communication network. Whether it be a home network or a business network, there is always the chance of the system could be exploited if not properly secured. An established network security system can help prevent the loss, theft or unauthorized access of sensitive information.




\section{Literature Review}
\section{Dataset}
\section{Implementation}
\section{Results}
\section{Conclusion}
\section*{Acknowledgment}

\begin{thebibliography}{00}
\end{thebibliography}

\end{document}
